%%%%%%%%%%%%%%%%%  Debut du fichier Latex  %%%%%%%%%%%%%%%%%%%%%%%%%%%%%%
\documentclass[12pt,onecolumn]{article}
%\usepackage[style=numeric,maxnames=1,uniquelist=false]{biblatex}
%\usepackage[backend=bibtex,style=numeric,minnames=4,maxnames=4,firstinits=true,sorting=none]{biblatex} 
\usepackage[backend=bibtex,bibstyle=phys,citestyle=authoryear,maxcitenames=1,minbibnames=3,maxbibnames=3,giveninits=true,natbib,doi=false,isbn=false]{biblatex} 
%\usepackage[authordate,bibencoding=auto,strict,backend=biber,natbib]{biblatex}

 %backend=biber is 'better'  
\makeatletter
\def\blx@maxline{77}
\makeatother
\renewbibmacro{in:}{} % to not have the "In:" to indicate the review
\AtEveryBibitem{\clearfield{title}} % to remove the titles in the biblio
% no page info
\AtEveryBibitem{%
  \ifentrytype{article}{%
    \clearfield{pages}%
  }{%
  }%
}
% no language info
\AtEveryBibitem{\clearlist{language}}
% no language no page
\AtEveryBibitem{%
  \clearfield{volume}%
  \clearfield{number}}
% To avoid parenthesis if no year entry in bib file
\renewbibmacro*{issue+date}{%
  \ifboolexpr{not test {\iffieldundef{year}} or not test {\iffieldundef{issue}}}
    {\printtext[parens]{%
       \iffieldundef{issue}
         {\usebibmacro{date}}
         {\printfield{issue}%
          \setunit*{\addspace}%
          \usebibmacro{date}}}}
    {}%
  \newunit}


\ExecuteBibliographyOptions{isbn=false,url=false,doi=false,eprint=false}

%\bibliography{/Users/Ileyk/Documents/Bibtex/Hubble_fellowship_no_url} 
\addbibresource{/Users/Ileyk/Documents/Bibtex/CNRS_19_fixed.bib}
%%% Pour un texte en francais


%%\usepackage[applemac]{inputenc}
%\usepackage[francais]{babel}
	         % encodage des lettres accentuees
\usepackage[T1]{fontenc}
\usepackage[utf8]{inputenc}          % encodage des lettres accentuees
%\usepackage{graphicx}
%%\usepackage{graphicx} \def\BIB{}
\usepackage[paper=a4paper,left=2.1cm,right=2.1cm,top=3.2cm,bottom=3.2cm]{geometry}
\usepackage{multicol}
\usepackage{graphicx,wrapfig,lipsum} 
%\def\BIB{}
\usepackage{caption}
\usepackage{subcaption}
\usepackage[pdftex]{hyperref}
%\usepackage{natbib}
\usepackage{url}
\usepackage{perpage} %the perpage package
\MakePerPage{footnote} %the perpage package command
\hypersetup{
    colorlinks,%
    citecolor=black,%
    filecolor=black,%
    linkcolor=black,%
    urlcolor=blue     % can put red here to visualize the links
}

\usepackage{enumitem}
\usepackage{amssymb}

%\renewcommand{\refname}{}

\usepackage{floatrow}

\usepackage{fancyhdr}
\usepackage{lastpage}

\pagestyle{fancy}
\fancyhf{}
\rhead{Research summary}
\lhead{El Mellah Ileyk}
\rfoot{\thepage / \pageref{LastPage}}

\DeclareUnicodeCharacter{00A0}{ }

\usepackage{xspace}

%%% Quelques raccourcis pour la mise en page
\newcommand{\remarque}[1]{{\small \it #1}}
\newcommand{\rubrique}{\bigskip \noindent $\bullet$ }
\newcommand{\sgx}{SgXB\xspace}
\newcommand{\sgxs}{SgXBs\xspace}
\newcommand{\ulx}{ULX\xspace}
\newcommand{\sfxt}{SFXT}
\newcommand{\sg}{Sg\xspace}
\newcommand{\co}{CO\xspace}
\newcommand{\gw}{GW\xspace}
\newcommand{\gws}{GWs\xspace}
\newcommand{\grb}{GRB\xspace}
\newcommand{\grbs}{GRBs\xspace}
\newcommand{\eos}{EOS\xspace}
\newcommand{\mhd}{MHD\xspace}
\newcommand*{\hmxb}{HMXB\@\xspace}
\newcommand*{\hmxbs}{HMXBs\@\xspace}
\newcommand*{\lmxb}{LMXB\@\xspace}
\newcommand*{\rlof}{RLOF\@\xspace}
\newcommand*{\ns}{NS\@\xspace}
\newcommand*{\nss}{NSs\@\xspace}
\newcommand*{\bh}{BH\@\xspace}
\newcommand*{\bhs}{BHs\@\xspace}
\newcommand*{\eg}{e.g.\@\xspace}
\newcommand*{\ie}{i.e.\@\xspace}
\newcommand*{\aka}{a.k.a. \@\xspace}
\newcommand*\diff{\mathop{}\!\mathrm{d}}
\newcommand{\mystar}{{\fontfamily{lmr}\selectfont$\star$}}
\newcommand*{\msun}{$M_{\odot}$\@\xspace}
\newcommand*{\mdotstar}{$\dot{M}_{\text{\mystar}}$\@\xspace}
\newcommand*{\mdotacc}{$\dot{M}_{\text{acc}}$\@\xspace}
\newcommand*{\ledd}{$L_{\text{Edd}}$\@\xspace}


\newcommand{\ignore}[1]{}

%\renewcommand*\rmdefault{iwona}

%\pagenumbering{gobble}

%\bibliographystyle{abbrvnat}
%\setcitestyle{authoryear,open={((},close={))}}

%\renewcommand{\thefootnote}{\roman{footnote}}

% -------------------------------------------------
\newcommand{\horrule}[1]{\rule{\linewidth}{#1}} % Create horizontal rule command with 1 argument of height

\title{	
\vspace*{-2.5cm}
%\normalfont \tiny 
%%\textsc{Paris Diderot} \\ [25pt] % Your university, school and/or department name(s)
%\horrule{0.5pt} \\[0.4cm] % Thin top horizontal rule
%\Large Speeding up the spinning top\\
%\large How accretion sets the pace in High Mass X-ray Binaries  \\ % The assignment title
%\horrule{2pt} \\[0.5cm] % Thick bottom horizontal rule
}
\author{\tiny} % Your name
\date{\tiny }%\normalsize\today} % Today's date or a custom date
% -------------------------------------------------

%\makeatletter
%\def\@xfootnote[#1]{%
%  \protected@xdef\@thefnmark{#1}%
%  \@footnotemark\@footnotetext}
%\makeatother

%\usepackage[square,numbers,sort]{natbib}
%\usepackage{har2nat} % "natbib" is loaded automatically

%
%\let\oldthebibliography\thebibliography
%\renewcommand{\thebibliography}[1]{%
%  \oldthebibliography{#1}
%  \let\oldbibitem\bibitem
%  \let\oldtextsc\textsc
%  \def\oldbbland{et}
%  \newcounter{authorcount}
%  \def\bibitem[##1]##2{%
%    \let\textsc\oldtextsc
%    \let\bbland\oldbbland
%    \oldbibitem[##1]{##2}%
%    \let\textsc\mytextsc%
%    \let\bbland\mybbland
%    \setcounter{authorcount}{0}
%  }
%  \def\mybbland{\setcounter{authorcount}{0}\oldbbland}
%  \def\dropetal##1.{ \bbletal}
%  \def\mytextsc##1{%
%    \oldtextsc{##1}%
%    \stepcounter{authorcount}%
%    \ifnum\value{authorcount}=2\relax%
%      \expandafter\dropetal%
%    \fi%
%  }%
%}


\begin{document}

%\bibpunct{[}{]}{;}{n}{,}{,}

%%%%%%%%%%%%%%%%%%%%%%%%%  PREMIERE PAGE %%%%%%%%%%%%%%%%%%%%%%%%%%%%%%
%%% DANS CETTE PAGE, ON REMPLACE LES INDICATIONS ENTRE CROCHETS [...]
%%% PAR LES INFORMATIONS DEMANDEES
%%%%%%%%%%%%%%%%%%%%%%%%%%%%%%%%%%%%%%%%%%%%%%%%%%%%%%%%%%%%%%%%%%%%%%%

\renewcommand{\headrulewidth}{1pt}
\pagestyle{fancy}
\fancyhf{}
\rhead{Tâche de service}
\lhead{El Mellah Ileyk}
\rfoot{\thepage / \pageref{LastPage}}

\begin{center}
\Large \textbf{Tâche de service SVOM/ECLAIRs}
\end{center}
\normalfont
\vspace*{-0.4cm}
\begin{table}[h!]
\centering
\label{my-label}
\begin{tabular}{|l|l|}
\hline
Type (ANO1 \`{a} ANO6) & ANO2 \& ANO3 \\ \hline
Nom du service & SO2 - Instrumentation spatiale \\ \hline
Nom de la t\^{a}che & SVOM/ECLAIRs \\ \hline
Labellisation & oui \\ \hline
Nom du responsable scientifique correspondant & Bertrand Cordier \\ \hline
Laboratoire et OSU dont elle rel\`eve & IRAP - OMP \\ \hline
\end{tabular}
\end{table}

%Tandis que les sursauts courts, d'une durée en gamma inférieure à 2 secondes, sont provoqués par la coalescence entre une étoile à neutron et une autre étoile à neutron ou un trou noir, les sursauts longs proviennent d'étoiles massives en fin de vie dont le c\oe{}ur s'effondre brusquement.

Les sursauts gamma comptent parmi les phénomènes les plus lumineux dans l'Univers. Observés à des distances cosmologiques depuis les années 60, ils sont probablement associés au lancement d'un jet ultra-relativiste depuis le voisinage immédiat d'un objet compact nouvellement formé. Outre qu'ils nous renseignent sur la première génération d'étoiles, les sursauts longs sondent l'Univers primordial et l'époque de la réionization et peuvent servir comme chandelles standards jusqu'à des distances bien plus importantes que les supernovae Ia, apportant ainsi de nouvelles contraintes sur les paramètres cosmologiques et la nature de l'énergie noire. Les sursauts courts ont connu un regain d'intérêt en 2017 après la détection d'une onde gravitationnelle produite par la coalescence de deux étoiles à neutron : pour la première fois, la détection s'est accompagnée d'une contrepartie photonique avec un sursaut gamma court suivi d'une kilonova. Les découvertes à venir grâce à cette nouvelle astronomie multi-messager sont considérables, parmi lesquelles la structure interne des étoiles à neutron et leur équation d'état, le mécanisme de formation des trous noirs et de lancement des jets ou encore la nucléosynthèse des éléments les plus riches en neutron.

Au cours des 15 dernières années, le satellite Swift a détecté près d'un millier de sursauts gamma. Cependant, seul un tiers d'entre eux ont pu voir leur distance mesurée. Avec la détection de nombreuses sources transitoires à venir grâce à LSST et SKA, un nouveau satellite dédié aux sursauts gamma et capable d'identifier précisément leur position est indispensable. \textit{Le satellite multi-longeurs d'onde sino-français SVOM} (\textit{Space Variable Objects Monitor}), dont le lancement est prévu pour 2021, sera consacré à l'observation des sursauts gamma. SVOM combine un détecteur gamma grand champ qui déclenche l'alerte et des instruments à plus basse énergie afin de localiser la source avant qu'elle ne s'évanouisse, d'identifier sa galaxie hôte et d'en déduire sa distance. La présente tâche de service porte essentiellement sur 2 composantes sous la responsabilité du FSC (\textit{French Scientific Center}) : \textit{le télescope grand champ X et gamma ECLAIRs} embarqué sur SVOM, semblable à l'instrument BAT sur Swift, et le segment sol associé, l'EIC (\textit{ECLAIRs Instrument Center}). Compte tenu de la date imminente du lancement, l'enjeu n'est plus tant la conception des instruments à bord que leur calibration. Dans le cadre de cette tâche de service, je souhaite réaliser la validation des performances d'ECLAIRs et caractériser sa réponse instrumentale afin d'optimiser tant le déclenchement des alertes que l'exploitation ultérieure des données par les observateurs. Au sein de l'EIC, je contribuerai au développement des fichiers de calibration et des outils de traitement des données SVOM mis à la disposition de la communauté.

\section{SVOM : charge utile et segment sol}

La mission SVOM rassemble un satellite éponyme et un segment sol composé d'un télescope de suivi au sol GFT-2 (\textit{Ground Follow-up Telescope}-2, ou Colibri) et d'un réseau d'antennes radio au sol VHF (\textit{Very High Frequency}) chargées de communiquer avec le satellite et d'orienter les télescopes au sol vers les sursauts gamma moins de 4 minutes après leur détection par ECLAIRs. Le satellite comporte plusieurs instruments :

\begin{itemize}
\setlength\itemsep{0em}
\item Développé par l'IRFU, l'IRAP et l'APC, \textbf{l'imageur grand champ X durs / gamma ECLAIRs} est doté d'un masque codé qui surplombe un plan focal de détection DPIX et sur lequel vient s'imprimer l'ombre du masque en cas de sursaut gamma. Ce dispositif confère à ECLAIRs à la fois un large champ de vision et une résolution angulaire suffisamment bonne pour apporter une localisation préliminaire de la source. Une unité de gestion et de traitement scientifique analyse le signal en temps réel, se charge de déclencher immédiatement l'alerte sursaut et de communiquer la position de la source aux autres instruments.
\item Le déclenchement d'une alerte sursaut provoque la rotation de SVOM afin de placer la source dans le champ de vision du \textbf{télescope X MXT} (\textit{Micro-channel X-ray Telescope}), bien plus réduit mais doté d'une meilleure résolution. % meilleure qu'une arcminute. Sa sensibilité et se résolution spectrale permettront de suivre le comportement spectroscopique de la source dans les X mous pendant la journée qui suivra l'alerte. 
\item En parallèle, \textbf{le télescope VT} (\textit{Visible Telescope}) apportera une localisation encore meilleure, de l'ordre de quelques arcsecondes dans le visible et proche infra rouge. %, et ce dans deux bandes spectrales dans le visible et proche infra rouge.
\item Enfin, \textbf{le GRM} (\textit{Gamma Ray Burst Monitor}) dont les 3 détecteurs permettront la localisation grossière de la source par triangulation dans un champ de vue plus large que celui d'ECLAIRs, en particulier pour les sursauts gamma courts. % L'objectif est d'apporter une aide précieuse à la recherche de contreparties photoniques aux ondes gravitationnelles.
\end{itemize}

\section{Contenu et équipe d'accueil}

L'un des objectifs majeur de l'IRAP est d'assurer le développement, la validation et l'exploitation de l'instrument ECLAIRs dont le laboratoire est PI avec son responsable scientifique Jean-Luc Atteia. A ces fins, 20 postes équivalent temps plein doivent être pourvus entre 2017 et 2022. Le support technique apporté par l'EIC, basé à l'IRAP et sous la responsabilité d'Olivier Godet, prend différentes formes : \textbf{phase préliminaire de commissionnement de l'instrument} (déploiement et étalonnage pour pallier à d'éventuels biais), maintien du logiciel de bord et \textbf{gestion des alarmes}, \textbf{suivi des opérations} et surveillance pour faire face aux éventuels problèmes instrumentaux qui pourraient survenir en vol. 

Au sein de cette équipe de 4 personnes, je pourrai dès 2019 participer au \textbf{développement d'outils d'analyse des données pour valider les performances de l'instrument} et mettre à la disposition de la communauté les fichiers de calibration qui garantiront une exploitation optimale une fois SVOM en activité. Pour ce faire, je prendrai comme référence des spectres simulés par le code Geant4 pour la source d'Américium-241 disponible à l'IRAP et les comparerai à ceux mesurés pour construire un modèle de la réponse spectrale qui sera utilisée au FSC par les observateurs pour ajuster les spectres construits à partir des données ECLAIRs. Je contribuerai à la \textbf{conception d'outils pour calculer certaines tables de configuration pour le bord} telles que les tables de seuils ASIC (\textit{Application-Specific Integrated Circuit}), de pixels bruyants/morts, d'efficacité, de gain ou d'offset.

\textbf{Le contrôle du bruit} introduit par les ASICs (\textit{Application-Specific Integrated Circuits}) du plan focal de détection DPIX est un enjeu majeur pour éviter que de fausses alertes ne soient déclenchées. J'évaluerai le niveau de bruit intrinsèque de chaque pixel mais aussi la capacité du bruit à se propager d'un pixel à l'autre en calculant la matrice de corrélations croisées du DPIX. J'identifierai des zones fiables homogènes où le bruit est faible et m'assurerai de la constance dans le temps de ces propriétés.

%Bonne homogénéité des détecteurs et des ASICs, mais quelques pourcents de pixels bruyants qui s'expriment au hasard dans le temps, de façon transitoire, et pourraient provoquer le déclenchement d'alertes faux-positives. Cross-talk propagates the noise. Zone sur le DPIX où les pixels montrent un bas niveau de bruit : contrôle de la constance dans le temps dans conditions de vol.

Pour peu qu'un sursaut soit suffisamment proche et brillant, la durée qui suit l'absorption d'un photon pendant lequel le détecteur ne peut plus absorber (ou temps mort) peut amener à sous-estimer le flux mesuré. Je modéliserai ce temps mort afin d'en évaluer l'impact sur les flux mesurés. En tant que membre de l'EIC, je développerai des scripts de tests pour vérifier le fonctionnement en vol du satellite, ainsi que des outils d'analyse pour en exploiter au mieux les données. 




%Target of Opportunity - Multi-Messenger (ToO-MM). Hebdomadaire. 
%Dans un second temps, les ToO occuperont une place plus importante dans la mission (jusqu'à 40\% de l'activité). ToO-MM avec Jean Jaubert (CNES, Toulouse) : Optimisation de la stratégie d'observation pour capturer un maximum de contreparties photoniques associées aux sursauts courts. Principalement avec MXT et VT. Simulations requises pour inclure toutes les contraintes relatives à la présence du Soleil, de la Lune ou de la Terre à proximité du champ de vision, au repositionnement des panneaux solaires, etc. Interface de requête ToO-MM. Méthode dite "\textit{smart}" pour optimiser les performances observationelles.



\section{Evolution de la tâche de service}

Dans la seconde phase de la mission SVOM, les \textit{Targets of Opportunity} (ToO) représenteront une part plus importante de l'activité du satellite. Je pourrais travailler à \textbf{l'optimisation de la présente stratégie d'observation} du MXT et du VT dite "\textit{smart}" pour capturer un maximum de contreparties photoniques associées aux sursauts courts, avec le \textit{ToO - Multi-Messenger} (ToO-MM, Jean Jaubert, CNES Toulouse), ainsi qu'à la \textbf{maintenance de l'interface de requête ToO-MM}.

A plus long terme, je souhaite mettre à profit l'expérience que j'aurai accumulé en rejoignant le projet SVOM pour contribuer à \textbf{l'instrument X-IFU} (\textit{X-ray Integral Field Unit}; PI: Didier Barret, IRAP) du satellite Athena (\textit{Advanced Telescope for High ENergy Astrophysics}) dont le lancement est prévu vers 2030. Après la fin prématurée de la mission Hitomi, le besoin pour un satellite X se fait plus que jamais sentir. La mission XRISM (\textit{X-Ray Imaging and Spectroscopy Mission}) de la Jaxa viendra temporairement pallier à ce manque dans les années 2020 et servira de démonstrateur à Athena.

%Bonus (CEA et APC) : MXT embarqués sur SVOM, le réseau d'alerte VHF, le télescope de suivi au sol GFT-2 (\textit{Ground Follow-up Telescope}). Je participerai à la vérification du bon fonctionnement du réseau VHF. Je prendrai part au contrôle du télescope franco-mexicain de suivi au sol GFT-2 dont la construction a été actée l'an dernier. 
%
%En complément, je pourrais travailler à l'optimisation de la présente stratégie d'observation du MXT et du VT dite "\textit{smart}" pour capturer un maximum de contreparties photoniques associées aux sursauts courts, avec le \textit{Target of Opportunity - Multi-Messenger} (ToO-MM, Jean Jaubert, CNES Toulouse), ainsi qu'à la conception de l'interface de requête ToO-MM.

\section{Compétences pour la tâche de service}

De par le contenu de mon activité de recherche (cf \textit{Research proposal}), je bénéficie d'une bonne connaissance des observables nécessaires à la validation des modèles à notre disposition. Les observations réalisées par SVOM pourront directement être confrontées avec les prédictions issues des simulations numériques d'accrétion sur des objets compacts que j'ai réalisées ces dernières années. Réciproquement, je serai à même d'aider à la définition de la priorité d'une ToO-MM et d'identifier les stratégies d'observations adéquates pour apporter de nouvelles contraintes à l'étude théorique des sursauts gamma, en particulier des sursauts courts en lien avec les détections à venir d'ondes gravitationnelles associées à la coalescence d'une étoile à neutron et d'un autre objet compact. En particulier, le sursaut gamma issu de la coalescence d'étoiles à neutron en 2017 est très atypique (sursaut faible, longue rémanence X, kilonova brillante) et sa nature reste encore incertaine.

Enfin, mon expertise numérique garantit que je pourrai être opérationnel dès mon affectation quel que soient les langages et conventions au sein de l'équipe d'accueil. Mes connaissances en hardware et en calculs de haute performance (e.g. architecture des super-calculateurs, méthodes de parallélisation massive et de multithreading) me permettront de développer les outils d'analyse en temps réel indispensables à la coordination des observations de sursauts gamma. 

%La culture technique 
%
%Colibri to observe the early optical afterglow durting the slew of the satellite
%To provide fast accurate positions of faint and dark GRBs and redshift estimator
%
%Polyvalent
%
%Agrégation
%
%co-PI XMM-Newton : collaboration avec l'équipe en charge du développement du MXT (PI: Diego G\"{o}tz, CEA)
%
%APC
%
%Expertise numérique :
%	- code
%	- hardware (HPC, parallélization)

\end{document}
%%%%%%%%%%%%%%%%%  Fin du fichier Latex  %%%%%%%%%%%%%%%%%%%%%%%%%%%%%%

